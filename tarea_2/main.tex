\documentclass{beamer}
\usetheme{Madrid}
\usecolortheme{default}
\usepackage[utf8]{inputenc}
\usepackage[spanish]{babel}
\usepackage{amsmath, amssymb}
\usepackage{graphicx}

\setbeamertemplate{headline}{
  \leavevmode%
  \hbox{%
  \begin{beamercolorbox}[wd=.5\paperwidth,ht=2.5ex,dp=1ex,left]{section in head/foot}%
    \hspace{1em}\textbf{Métodos de Optimización}
  \end{beamercolorbox}%
  \begin{beamercolorbox}[wd=.5\paperwidth,ht=2.5ex,dp=1ex,right]{section in head/foot}%
    \textbf{Universidad Nacional del Altiplano - FINESI}\hspace{1em}
  \end{beamercolorbox}}%
}

\setbeamertemplate{footline}[frame number]

\title{Universidad Nacional del Altiplano Puno}
\subtitle{Escuela Profesional de Ingeniería Estadística e Informática}

\author{
  \textbf{Resolución de Sistemas de Ecuaciones Lineales con Python} \\
  Usando Métodos de Sustitución, Igualación y Reducción \\
  \vspace{0.5cm}
  \textbf{Integrantes:} \\
  Beatriz Umiña Machaca \\
  Belinda Apaza Quispe \\
  \\
  \textbf{Docente:} \\
  Fred Torres Cruz
}

\institute{}

\date{16 de abril de 2025}


\begin{document}
\frame{\titlepage}


\section{Sistema de Ecuaciones}
\begin{frame}{¿Qué es un Sistema de Ecuaciones?}
  \begin{itemize}
    \item Un \textbf{sistema de ecuaciones} es un conjunto de dos o más ecuaciones con varias incognitas.
    \item El objetivo es encontrar los valores de esas variables.
  \end{itemize}
  
  \vspace{0.3cm}
  \textbf{Ejemplo:}
  \[
    \begin{cases}
      2x + y = 5 \\
      x - y = 1
    \end{cases}
  \]

  \vspace{0.3cm}
  \textbf{Métodos comunes para resolverlos:}
  \begin{itemize}
    \item \textbf{Sustitución:} Se despeja una variable en una ecuación y se sustituye en la otra.
    \item \textbf{Igualación:} Se despeja la misma variable en ambas ecuaciones y se igualan.
    \item \textbf{Reducción:} Se suman o restan las ecuaciones para eliminar una variable.
  \end{itemize}
\end{frame}


\section{Objetivo}
\begin{frame}{Objetivo del Programa}
  \textbf{Objetivo:} \\
  Crear un programa que resuelva un sistema de dos ecuaciones lineales con dos incógnitas \(x\) y \(y\), utilizando tres métodos algebraicos:
  \begin{itemize}
    \item \textbf{Sustitución:} Se despeja una variable de una ecuación y se sustituye en la otra.
    \item \textbf{Igualación:} Se despejan las mismas variables en ambas ecuaciones y se igualan.
    \item \textbf{Reducción:} Se suman o restan las ecuaciones para eliminar una variable.
  \end{itemize}
\end{frame}

\section{Lógica del Programa}
\begin{frame}{Lógica del Programa}
  \begin{itemize}
    \item Ingreso de ecuaciones por el usuario.
    \item Conversión de ecuaciones de texto a expresión algebraica.
    \item Menú con 3 métodos para resolver el sistema.
    \item Cálculo y visualización de resultados.
  \end{itemize}
\end{frame}

\section{Librerías}
\begin{frame}{Librerías Usadas}
  \texttt{import re} \\
  \texttt{from sympy import symbols, Eq, solve, sympify}
  \vspace{0.5cm}
  \begin{itemize}
    \item \textbf{re:} Para manipular texto (expresiones regulares).
    \item \textbf{sympy:} Librería simbólica para resolver ecuaciones, trabajar con símbolos, etc.
  \end{itemize}
\end{frame}

\section{Parseo de Ecuaciones}
\begin{frame}{Ingreso y Parseo de Ecuaciones}
  \texttt{def parsear\_ecuacion(ecuacion\_str):}
  \begin{itemize}
    \item Elimina espacios innecesarios.
    \item Usa expresiones regulares para separar la ecuación.
    \item Convierte cadenas de texto en expresiones simbólicas.
  \end{itemize}
\end{frame}

\section{Métodos para Resolver}
\begin{frame}{Método de Sustitución}
  \textbf{Pasos:}
  \begin{enumerate}
    \item Se despeja una variable (x o y) de una de las ecuaciones.
    \item Se sustituye en la otra ecuación.
    \item Se resuelve la ecuación resultante y luego se halla la otra variable.
  \end{enumerate}
\end{frame}

\begin{frame}{Método de Igualación}
  \textbf{Pasos:}
  \begin{enumerate}
    \item Se despeja la misma variable en ambas ecuaciones.
    \item Se igualan las dos expresiones resultantes.
    \item Se resuelve la ecuación para una variable, luego la otra.
  \end{enumerate}
\end{frame}

\begin{frame}{Método de Reducción}
  \textbf{Pasos:}
  \begin{enumerate}
    \item Se multiplican las ecuaciones si es necesario para igualar coeficientes.
    \item Se suman o restan las ecuaciones para eliminar una variable.
    \item Se resuelve y se halla la variable faltante.
  \end{enumerate}
\end{frame}

\section{Menú Interactivo}
\begin{frame}{Menú Interactivo en Python}
\begin{itemize}
  \item Menú con opciones 1, 2, 3 para elegir método.
  \item Opción 4 para salir del programa.
  \item El usuario puede repetir el proceso las veces que desee.
\end{itemize}
\end{frame}

\section{Ejemplo de Uso}
\begin{frame}{Ejemplo de Ejecución}
\textbf{Ecuaciones:}
\begin{itemize}
  \item \(2x + y = 5\)
  \item \(x - y = 1\)
\end{itemize}
\textbf{Resultado esperado:}
\begin{itemize}
  \item Se puede resolver con cualquiera de los tres métodos.
  \item Muestra \(x = 2\), \(y = 1\)
\end{itemize}
\end{frame}

\section{Conclusiones}
\begin{frame}{Conclusiones}
\begin{itemize}
  \item El programa permite resolver sistemas lineales de forma automática.
  \item Es interactivo y didáctico.
  \item Útil para estudiantes que aprenden álgebra y programación.
\end{itemize}
\end{frame}

\end{document}